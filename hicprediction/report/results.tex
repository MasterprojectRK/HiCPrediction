\section{Results} \label{sec:results:main}
In the following subsections, the results for the changes to hicprediction introduced in 
\autoref{sec:improve:main} will be presented and interpreted. 
Furthermore, \autoref{sec:res:compare} provides a direct comparison 
of the (improved) results from hicprediction to the ones from HiC-Reg.

\subsection{Avoiding training samples without protein data} \label{sec:res:removeEmptySamples}
\begin{wrapfigure}{r}{0.5\textwidth}
 \centering
 \resizebox{0.5\textwidth}{!}{
 \scriptsize
 \import{figures/}{pearson_emptyVsNonempty_narrowPeak.pdf_tex}}
 \caption{Pearson correlation before/after removing samples}
 \label{fig:results:emptyVsNonempty:Pearson}
\end{wrapfigure}
The effect of removing empty samples from the datasets was clearly visible,
both in terms of Pearson correlation and matrix plots, 
see \autoref{fig:results:emptyVsNonempty:Pearson} and \ref{fig:results:pk:noEmpty:5k}.
It is obvious that inverted triangles and gradient-style predictions were removed by discarding empty samples,
but the prediction is also strongly fragmented, so that structures like TADs and loops are hardly visible in the plots.
For example, in GM12878 and K562, only about 25\% of the originally \num{3227900} samples remained in the datasets,
which means that 75\% of all pairs in the predicted Hi-C matrix for K562 had the default interaction count zero.
This effect would be even more striking for the explanation example from \autoref{sec:improve:emptySamples},
where only 6 of \num{79900} samples would remain in the dataset. 
In this (staged) case, the algorithm then just ``learns'' the relation 
``sample exists $\Rightarrow$ read count is 100'',
without any of the features (start, end, window, distance) being considered as a decision criterion.
\begin{figure}[h]
 \centering
 \scriptsize
 \import{figures/}{trianglesGradientsAfter2.pdf_tex}
 \caption{Example prediction, without empty samples}
 \label{fig:results:pk:noEmpty:5k}
\end{figure}

\subsection{Dealing with sparse input data} \label{sec:res:dealing_with_sparse_input}
Filtering fragmented output matrices like the ones obtained in \autoref{sec:res:removeEmptySamples} 
was unsuccessful, since no parametrization of the 2D-Gaussian filters proved useful -- either the gaps were not 
filled, or the structures in the matrix were
blurred too much, or both, see \autoref{fig:results:pk:smoothened:5k}.
Other kernel types and more sophisticated image-inpainting methods \cite{Elharrouss2019}
might do better, but choosing the appropriate ones and adapting them to the problem at hand would be 
laborious and beyond the scope of this masterproject.
Additionally, and even more important, it is not clear whether such sparse matrices as depicted in
\autoref{fig:results:pk:noEmpty:5k} would be a good starting point for suchlike approaches. 
The missing regions can be several \num{100000}s of base pairs wide (e.\,g. \autoref{fig:results:pk:noEmpty:5k}, upper right),
and both the low Pearson correlation (\autoref{fig:results:emptyVsNonempty:Pearson}) and the lack of structure in the predicted matrices
(\autoref{fig:results:pk:noEmpty:5k}) suggest there might simply not be enough information 
for imputing missing values, with whatever approach.

Filtering the inputs with one-dimensional Gaussian filters was more successful.
While it seems impossible to fill long stretches of empty bins by Gaussian smoothing alone, some structures became quite well visible
and the Pearson correlation also improved slightly, \autoref{fig:results:pk:sigma4p0:5k} and \ref{fig:results:pk:Pearson:sigma4p0:5k}.
Apart from the gaps remaining in the prediction, smoothing the proteins also has the disadvantage of blurring and shifting the structures in 
the predictions compared to the target matrices.
This is typical for Gauss filters, but probably acceptable in this case.

In predictions from bigwig files, there were much less empty bins and no inverted triangles or gradient-style regions, 
\autoref{fig:results:bigwig:allsamples} and \autoref{fig:app:bw:others:5k} to \ref{fig:app:Pearson:furtherCellLines} 
on page \pageref{fig:app:bw:others:5k}\,ff.
Due to the higher peak density in the bigwig files, the effect of leaving out 
empty samples was sometimes hardly recognizable, compare \autoref{fig:results:bigwig:allsamples} and \ref{fig:results:bigwig:noemptysamples}.
The Pearson correlations were also similar for both approaches, and both were better than the ones from using peak files, 
\autoref{fig:results:bigwig:Pearson}.
However, the effect of removing samples was found to be dependent on cell line, chromosome and genomic position. 
Some chromosomes, e.\,g. 21, have large contiguous regions which seem to be deprived of all proteins,
probably for biological reasons.
For such regions, removing empty samples was significantly better, \autoref{fig:app:GM12878:K562:chr21:25k:goodToRemove}, 
but detrimental for other regions of the same chromosome, see \autoref{fig:app:GM12878:K562:chr21:25k:badToRemove}.
This was unfortunately not reflected in the Pearson correlation, \autoref{fig:app:Pearson:GM12878:K562:chr21:25kb}.
Note that the area under the correlation curve is up to 0.7, which is a surprisingly high value considering
the modest look of the matrix plots, \autoref{fig:app:GM12878:K562:chr21:25k:goodToRemove} 
and \ref{fig:app:GM12878:K562:chr21:25k:badToRemove}.

With regard to background signal and sequencing noise in the bigwig files, 
the results from chromosome 17 suggest that the algorithm can at least partially 
learn which ChIP-seq read count values are indicative of high interaction counts and which are not. 
Beyond that, using a threshold to set signal values below a certain level to zero is possible,
but finding a value which works well across all cell lines, chromosomes and proteins seems challenging.
Too large values again lead to sparsity and too small values
do not have noteworthy influence on the result.

Avoiding empty bins in the inputs by increasing the input matrix resolution (bin size) to 25\,kbp 
worked surprisingly well, even without further measures, \autoref{fig:results:pk:allsamples:25k}.
Especially larger structures like the ones from 12...\SI{13}{\mega\bp} and 13...\SI{14}{\mega\bp} were usually at least signified.
However, gradient-style regions and inverted triangles triangles still occurred, albeit to a lesser extent, 
since the underlying problem remains the same, see e.\,g. the regions around 13 and \SI{14.5}{\mega\bp}
in \autoref{fig:results:pk:allsamples:25k}.

As expected, discarding empty samples led to less fragmentation with 25\,kbp than with 5\,kbp,
because fewer input bins were empty; the result was still not useful, \autoref{fig:results:pk:noemptysamples:25k}.
Again, it was not possible to fill in missing values with a Gauss filter without blurring the
matrix too much, \autoref{fig:results:pk:smoothened:25k}.
For 25\,kbp resolution, too, smoothing the proteins worked better than smoothing the predicted matrix. 
Larger gaps could again not be closed, but apart from that, structures were often more easy to identify than 
without smoothing, compare \autoref{fig:results:pk:allsamples:25k} and \ref{fig:results:pk:proteinsSmoothened:25k}.

Predictions using bigwig files were also performed, 
but at 25\,kbp resolution, it was even more difficult to say whether these were better or worse than the ones from peak files,
compare \autoref{fig:results:bigwig:allsamples:25k} and \ref{fig:results:pk:allsamples:25k}.
As already mentioned above, it cannot generally be said whether leaving out or keeping empty samples is better.
In the example matrix snippet from chromosome 17, it was again difficult to determine differences, 
\autoref{fig:results:bigwig:allsamples:25k}, \ref{fig:results:bigwig:noemptysamples:25k} 
and \ref{fig:results:bigwig:Pearson25k}, but this might not hold for all regions.

In terms of Pearson correlation, all predictions from 25\,kbp resolution were better
than the best one from \SI{5}{\kilo\bp}, \autoref{fig:results:bigwig:Pearson25k}, and structures indeed sometimes seemed 
more recognizable in the plots for \SI{25}{\kilo\bp}. 
Interestingly, the fragmented prediction from peak files without empty samples, \autoref{fig:results:pk:noemptysamples:25k},
showed one of the best Pearson correlations obtained thus far.
This is probably because interaction counts of discarded regions are set to zero, 
which seems to be correct in many cases, especially in the distance range [0.4...1.0]\,Mbp.
Independent of the root cause, this example again underscores that relying on Pearson correlation alone is unfavorable,
because the corresponding matrices can be useless in practice.

For \SI{25}{\kilo\bp}, predictions were also performed using bigwig- and peak files simultaneously,
thus increasing the number of features in the samples from $3\cdot12+1=37$ to $3\cdot24+1=73$ (2x12 start-, 2x12 window- and 2x12 end features plus 
distance).
While the Pearson correlations were mostly better than with bigwig files alone, especially when keeping empty 
samples, \autoref{fig:app:Pearson:GM12878:K562:chr17:25kb:combi} and \ref{fig:app:Pearson:GM12878:K562:chr21:25kb:combi},
not much improvement was visible in the predicted matrices, \autoref{fig:results:combined:chr17:25k}, \ref{fig:app:GM12878:K562:chr21:25k:combiFew} and 
\ref{fig:app:GM12878:K562:chr21:25k:combiMany}.
Feature importance plots show that the influence of the start- and end features from the peak files was generally negligible, 
but the window features were often considered more important than most features from 
bigwig files, \autoref{fig:chr17:GM12878:K562:combined:featImport} and \ref{fig:chr21:GM12878:K562:combined:featImport}.
\begin{figure}[hp]
 \centering
 \scriptsize
 \import{figures/}{chr17-12-15_pk_GM12878-on-K562_5k_sigma5p0.pdf_tex}
 \caption{Example prediction from peaks, no empty samples, 5\,kbp, matrix filtered $\sigma=5.0$}
 \label{fig:results:pk:smoothened:5k}
\end{figure}
\begin{figure}[hp]
 \centering
 \scriptsize
 \import{figures/}{chr17-12-15_pk_GM12878-on-K562_5k_smooth4p0.pdf_tex}
 \caption{Example prediction from peaks, no empty samples, 5\,kbp, proteins filtered $\sigma=4.0$}
 \label{fig:results:pk:sigma4p0:5k}
\end{figure}
\begin{figure}[hp]
 \centering
 \scriptsize
 \import{figures/}{pearson_GM12878-on-K562_5k_smooth4p0.pdf_tex}
 \caption{Pearson correlation, 5\,kbp, proteins filtered $\sigma=4.0$}
 \label{fig:results:pk:Pearson:sigma4p0:5k}
\end{figure}
\begin{figure}[hp]
 \centering
 \scriptsize
 \import{figures/}{chr17-12-15_bigwig_GM12878-on-K562.pdf_tex}
 \caption{Example prediction from bigwig, all samples}
 \label{fig:results:bigwig:allsamples}
\end{figure}
\begin{figure}[hp]
 \centering
 \scriptsize
 \import{figures/}{chr17-12-15_bigwig_removeEmpty_GM12878-on-K562.pdf_tex}
 \caption{Example prediction from bigwig, no empty samples}
 \label{fig:results:bigwig:noemptysamples}
\end{figure}
\begin{figure}[hp]
 \centering
 \scriptsize
 \import{figures/}{pearson_emptyVsNonempty_bigwig-GM12878-on-K562.pdf_tex}
 \caption{Pearson correlation before/after removing samples, 5\,kbp}
 \label{fig:results:bigwig:Pearson}
\end{figure}
\begin{figure}[hp]
 \centering
 \scriptsize
 \import{figures/}{chr17-12-15_pk_GM12878-on-K562_25k_noRemoveEmpty.pdf_tex}
 \caption{Example prediction from narrow-/broadPeak, 25\,kbp, all samples}
 \label{fig:results:pk:allsamples:25k}
\end{figure}
\begin{figure}[hp]
 \centering
 \scriptsize
 \import{figures/}{chr17-12-15_pk_GM12878-on-K562_25k_removeEmpty.pdf_tex}
 \caption{Example prediction from peaks, no empty samples, 25\,kbp}
 \label{fig:results:pk:noemptysamples:25k}
\end{figure}
\begin{figure}[hp]
 \centering
 \scriptsize
 \import{figures/}{chr17-12-15_pk_GM12878-on-K562_25k_smoothMatrix1p0.pdf_tex}
 \caption{Example prediction from peaks, no empty samples, 25\,kbp, matrix filtered $\sigma=1.0$}
 \label{fig:results:pk:smoothened:25k}
\end{figure}
\begin{figure}[hp]
 \centering
 \scriptsize
 \import{figures/}{chr17-12-15_pk_GM12878-on-K562_25k_smoothProteins1p0.pdf_tex}
 \caption{Example prediction from peaks, no empty samples, \SI{25}{\kilo\bp}, proteins filtered $\sigma=1.0$}
 \label{fig:results:pk:proteinsSmoothened:25k}
\end{figure}
\begin{figure}[hp]
 \tiny
 \centering
 \import{figures/}{GM12878-on-K562_chr17_featImportances_combined.pdf_tex}
 \caption{feature importances, GM12878 on K562, chr17, bigwig- and peak files combined}
 \label{fig:chr17:GM12878:K562:combined:featImport}
\end{figure}
\begin{figure}[hp]
 \tiny
 \centering
 \import{figures/}{GM12878-on-K562_chr21_featImportances_combined.pdf_tex}
 \caption{feature importances, GM12878 on K562, chr21, bigwig- and peak files combined}
 \label{fig:chr21:GM12878:K562:combined:featImport}
\end{figure}
\begin{figure}[hp]
 \centering
 \scriptsize
 \import{figures/}{chr17-12-15-bw-GM12878-on-K562-25k_noRemoveEmpty.pdf_tex}
 \caption{Example prediction from bigwig, \SI{25}{\kilo\bp}, all samples}
 \label{fig:results:bigwig:allsamples:25k}
\end{figure}
\begin{figure}[hp]
 \centering
 \scriptsize
 \import{figures/}{chr17-12-15_bw_GM12878-on-K562_25k_removeEmpty.pdf_tex}
 \caption{Example prediction from bigwig, no empty samples, \SI{25}{\kilo\bp}}
 \label{fig:results:bigwig:noemptysamples:25k}
\end{figure}
\begin{figure}[hp]
 \centering
 \scriptsize
 \import{figures/}{pearson_GM12878-on-K562_25k.pdf_tex}
 \caption{Pearson correlation comparison}
 \label{fig:results:bigwig:Pearson25k}
\end{figure}
\begin{figure}[hp]
 \centering
 \scriptsize
 \import{figures/}{chr17-12-15_GM12878-on-K562_25kb_combi.pdf_tex}
 \caption{Example combined prediction, \SI{25}{\kilo\bp}}
 \label{fig:results:combined:chr17:25k}
\end{figure}

\subsection{Scaling protein signal values and Hi-C interaction counts} \label{sec:res:normalization}
For three of four tested scales for Hi-C interaction counts, [0..10], [0...100] and [0...1000],
predicted and (scaled) target value were in the desired order of magnitude and 
the Pearson correlation was very similar to the status quo, 
see \autoref{fig:app:Pearson:GM12878:K562:chr17:5kb:normalized} and 
\ref{fig:app:GM12878:K562:chr17:normalizedMatrices}.
For value range [0...1], the Pearson correlation was notably worse than without scaling, 
at least for distances greater than \SI{0.2}{\mega\bp}.
No obvious reason for this could be found in the plots of the predicted matrices, 
but the usual ``logarithm plus 1''-representation is also not well suited 
for matrices with interaction counts generally smaller than 1. 
The target matrices, too, do not look very informative when scaled and plotted this way,
which is a reason on its own to go without the [0...1] range.
\begin{figure}[hb]
 \centering
 \begin{subfigure}{.495\textwidth}
  \centering
  \resizebox{\textwidth}{!}{
  \scriptsize
  \import{figures/}{pearson_GM12878-on-K562_normalized.pdf_tex}}
  \caption{matrices scaled}
  \label{fig:app:Pearson:GM12878:K562:chr17:5kb:normalized}
 \end{subfigure}\hfill
\begin{subfigure}{.495\textwidth}
  \centering
  \resizebox{\textwidth}{!}{
  \scriptsize
  \import{figures/}{pearson_GM12878-on-K562_protsNormalized.pdf_tex}}
  \caption{proteins scaled}
  \label{fig:app:Pearson:GM12878:K562:chr17:5kb:protsNormalized}
 \end{subfigure}
 \caption{Pearson correlations, matrices / proteins scaled}
 \label{fig:res:Pearson:matrices:proteins:normalized:5kb}
\end{figure}

Feature scaling yielded the expected outcome in terms of Pearson correlations,
\autoref{fig:app:Pearson:GM12878:K562:chr17:5kb:protsNormalized}, i.\,e. all value ranges
had similar performance and were no better or worse than without.
In terms of Hi-C matrices, the results were surprisingly worse than without scaling,
with errors in the predictions e.\,g. around 13.5 and \SI{14.0}{\mega\bp},  
see \autoref{fig:app:GM12878:K562:chr17:normalizedProteins}.

The comparatively bad outcome might partially be due to the small set-to-zero threshold
which was applied after adjusting the value range, \autoref{sec:methods:normalization}.
As this is a non-continuous transformation of the value range, thresholding could in principle affect the splitting.
However, it would be very surprising if this was the main cause, 
as only very few values fell below the threshold for each protein; 
for example, in value range [0...10], at most 53, or 0.3\% of \num{16240} samples from a single protein were set to zero. 
Additionally, the threshold hypothesis does not explain the errors in predictions with value range [0...1],
where the threshold was not applied.
On the other hand, it is also hard to imagine that 
the errors in the predictions occurred by chance, as they were approximately at the same positions across all
concerned predictions.

To investigate further on the issue, predictions were also made for \SI{25}{\kilo\bp} resolution,
Here, too, the outcome was partially unexpected.
While the Pearson correlations indicated a positive influence of scaling for all but
the division-by-mean method, \autoref{fig:app:Pearson:GM12878:K562:chr17:protNorms:25kb},
the plots of the matrices looked very similar to the non-scaled state, 
\autoref{fig:app:GM12878:K562:chr17:normalizedProteins:25k}.
The reason for this behavior could unfortunately not be determined.

Because scaling input features does not seem very common for random forests, only a single paper was
found on the topic.
Here, Dinc. et al. investigated the influence of value-range (min-max) scaling and z-score normalization\footnote{
subtract mean and divide by standard deviation} 
on the performance of several classifiers for protein crystallization images \cite{Dinc2014}.
For the random forest classifier, some improvements for the z-score normalization,
but only very little change for value-range scaling were recorded. 
However, the reasons for the improvement were not investigated.
The already mentioned study by Strobl et colleagues \cite{Strobl2007} does not cover feature scaling
directly, but found that features with different value ranges can cause bias in variable 
selection, especially when used in combination with categorical variables and bootstrapping with replacement.

Within this masterproject, it could not be clarified what exactly caused the
unexpected results when scaling feature value ranges.
However, first tests showed an interesting relation between the
floating point precision of the feature values and the predicted matrices,
which could be investigated in future studies.
Currently, hicprediction is using 32-bit floating point numbers rounded
to 6 digits after the decimal point -- and this may be insufficient
for small value ranges, considering the strongly nonuniform interaction count distribution, 
\autoref{fig:GM12878:K562:interactionCountDistribution}.
For example, when using value range [0...1] and \SI{5}{\kilo\bp} matrix resolution, 
there are \num{100000} different floating point numbers in the 
interval [0,~0.100000) after rounding to six digits after the decimal point -- but more than half of the approximately 3.2 million samples have interaction 
counts lying in this interval.

Since protein scaling yielded unfavorable results,
the remaining computations were performed without.

\subsection{Concatenating datasets and predictions from different cell lines}\label{sec:res:concat}
In order to combine predictions from different cell lines on K562, 
the single predictions were computed first, \autoref{fig:app:all:K562:chr17:singleVsJoint:5k},
top 4 panels.
It is obvious that not all cell lines  are equally well suited for predicting K562,
which is probably due to different biological functions.
Although e.\,g. the prediction from HUVEC showed only fairly few structures,
the averaged prediction from GM12878, HMEC, HUVEC and NHEK on K562 was still acceptable,
partially even better than the best single-cell-line prediction, 
\autoref{fig:res:Pearson:singleVsTogether:K562:chr17:5k} and 
\ref{fig:app:all:K562:chr17:singleVsJoint:5k}. 
Both the Pearson correlation and the plot of the predicted matrix look more smooth than the ones
from single cell lines, which is due to the averaging process and thus not surprising.
The structures in the matrix plot were still distinguishable and not significantly worse
than the ones from single-cell-line predictions.

The prediction from concatenated datasets performed not as well as the averaged single predictions,
but still better than the single predictions from ``less suitable'' cell lines. 
It should however be noted that the datasets have been concatenated without prior feature scaling, due 
to the problems mentioned in \autoref{sec:res:normalization} above.
This is probably suboptimal, because the same features can have significantly different value ranges in different cell lines,
see \autoref{fig:app:prots:valueRangeBefore}.
The predictions from concatenated datasets might thus improve once the 
existing feature scaling problems have been investigated and resolved.


\begin{figure}
\centering
\begin{subfigure}{.495\textwidth}
  \centering
  \resizebox{\textwidth}{!}{
  \scriptsize
 \import{figures/}{pearson_allSingle-on-K562_chr17_5k.pdf_tex}}
 \caption{single predictions}
 \label{fig:res:Pearson:allSingle:K562:chr17:5kb}
\end{subfigure}\hfill%
\begin{subfigure}{.495\textwidth}
  \centering
  \resizebox{\textwidth}{!}{
  \scriptsize
 \import{figures/}{pearson_allTogether-on-K562_chr17_5k.pdf_tex}}
 \caption{joint predictions}
 \label{fig:res:Pearson:allTogether:K562:chr17:5kb}
\end{subfigure}
\caption{Pearson correlations, single vs. joint predictions on K562}
\label{fig:res:Pearson:singleVsTogether:K562:chr17:5k}
\end{figure}



\subsection{Emphasizing certain samples} \label{sec:res:emphasizing}
After 200 runs of the tree-structured Parzen estimator, see \autoref{sec:methods:emphasizing},
the supposedly best parameters for weighting samples according to interaction count were determined as 
$lower=292.635$, $upper=857.073$ and $k=14.625$. 
This means that all samples with an interaction count in the range [292.635...857.073] were given integer weights such that 
the sum of the weighted samples was approximately 14.625 times the weight sum (=number) of the unweighted samples.
Unfortunately, there were only six samples in the given range, 
so the effect of interaction-count-based weighting on the predictions was quite small.
Both the resulting Pearson correlation and the matrix plots looked fairly similar for predictions with weighted samples
and usual predictions, \autoref{fig:res:Pearson:K562:chr17:5kb:weightingInteractionCts} 
and \ref{fig:app:GM12878:K562:chr17:interactionCountWeighting:matrices}.

With another 200 runs of the estimator, the supposedly best parameters for weighting samples according to CTCF signal value
were determined as $lower=228.8$, $upper=670.112$ and $k=4.241$. 
In this case, there were \num{285242} samples with CTCF start-, end- or window-feature in the given range, 
so an influence of CTCF-based weighting on the results was to be expected, which also precipitated in the feature importances,
\autoref{fig:app:GM12878:K562:chr17:CTCFWeighting:importances}. 
Interestingly, only the CTCF-window feature gained importance, while start- and end-feature 
remained at low levels.
However, the Pearson correlation of the prediction from weighted samples was actually worse than before, 
\autoref{fig:res:Pearson:K562:chr17:5kb:weightingCTCF},
and the matrix plots showed at least no significant improvement, fig\,\ref{fig:app:GM12878:K562:chr17:CTCFWeighting:matrices}.

Removing all samples with a distance below \SI{5000}{\bp} surprisingly improved and also slightly smoothened the Pearson correlation, 
\autoref{fig:res:Pearson:K562:chr17:5kb:noDiags}.
The matrix plots also clearly differed from the standard predictions, but it is difficult to say whether they are better,
\autoref{fig:app:GM12878:K562:chr17:noDiagonal}.
While some predicted structures were more distinct, there were also some which do not seem to match any real interacting regions, 
and the contrast between interacting- and non-interacting regions also seemed lower than in the standard predictions.

TAD-based weighting of samples did not change the prediction results much.
Both the Pearson correlation and the resulting matrices were quite similar to the ones from standard predictions,
\autoref{fig:res:Pearson:K562:chr17:5kb:weightingTADs} and \ref{fig:app:GM12878:K562:chr17:TadWeighting}.
Note that for weighting factor $k=0.1$ the results were identical to the the status quo,
because it turned out that $\frac{\sum_{weightedSamples}weight}{\sum_{unweightedSamples}weight}\approx 0.1$, so no weighting was performed at all
due to rounding (\autoref{sec:methods:emphasizing}).


\begin{figure}
\centering
\begin{subfigure}{.495\textwidth}
  \centering
  \resizebox{\textwidth}{!}{
  \scriptsize
 \import{figures/}{pearson_GM12878-on-K562_chr17_readCountWeighting.pdf_tex}}
 \caption{weighting according to interaction counts}
 \label{fig:res:Pearson:K562:chr17:5kb:weightingInteractionCts}
\end{subfigure}\hfill%
\begin{subfigure}{.495\textwidth}
  \centering
  \resizebox{\textwidth}{!}{
  \scriptsize
 \import{figures/}{pearson_GM12878-on-K562_chr17_CTCFweighting.pdf_tex}}
 \caption{weighting according to CTCF signal value}
 \label{fig:res:Pearson:K562:chr17:5kb:weightingCTCF}
\end{subfigure}

\vspace{5mm}
\begin{subfigure}{.495\textwidth}
  \centering
  \resizebox{\textwidth}{!}{
  \scriptsize
 \import{figures/}{pearson_GM12878-on-K562_chr17_noDiagonal.pdf_tex}}
 \caption{removing samples with distance $\leq$ \SI{5}{\kilo\bp}}
 \label{fig:res:Pearson:K562:chr17:5kb:noDiags}
\end{subfigure}
\begin{subfigure}{.495\textwidth}
  \centering
  \resizebox{\textwidth}{!}{
  \scriptsize
 \import{figures/}{pearson_GM12878-on-K562_5k_TADweighting.pdf_tex}}
 \caption{weighting according to TADs}
 \label{fig:res:Pearson:K562:chr17:5kb:weightingTADs}
\end{subfigure}
\caption{Pearson correlations for different sample weighting approaches}
\label{fig:res:Pearson:sampleWeighting:5k}
\end{figure}

\subsection{Replacing random forest by extra tree regression} \label{sec:res:extratrees}
Predictions using the extra trees algorithm were generally similar to predictions from random forests.
While Pearson correlations were in favor of the random forest algorithm, \autoref{fig:res:Pearson:extratreesVsForests},
the matrix plots looked fairly similar, \autoref{fig:app:GM12878:K562:chr17:forestVsExtratrees}.
It remains for future research whether this holds for other cell lines and chromosomes as well.
\begin{figure}[hb]
  \centering
  \scriptsize
  \import{figures/}{pearson_GM12878-on-K562_extraTrees_vs_Forests.pdf_tex}
  \caption{Pearson correlations extra trees vs. random forest}
  \label{fig:res:Pearson:extratreesVsForests}
\end{figure}

\subsection{Comparison between HiC-Reg and hicprediction} \label{sec:res:compare}
Looking at the matrices obtained with hicprediction thus far, it was obvious that -- despite some improvements compared to the original state -- 
the predictions were still inferior to the ones published for HiC-Reg.
\autoref{fig:res:HicregVsHicprediction_supplementary} shows a direct juxtaposition between a cross-cell prediction from HiC-Reg -- 
reconstructed from data published along with the article \cite{Zhang2019} --
and the corresponding prediction from hicprediction.
In terms of Pearson correlation, the results from hicprediction and HiC-Reg were very different, 
see \autoref{fig:res:Pearson:K562:chr17:5kb:HicregFromPublication}. 
While hicprediction was sometimes better at smaller distances, in the given test case particularly obvious below \SI{0.5}{\mega\bp}, 
the published results from HiC-Reg were always -- not only in the test case plotted in \autoref{fig:res:Pearson:K562:chr17:5kb:HicregFromPublication} -- better 
for larger distances.

To check whether the comparatively worse results from hicprediction where due to the underlying algorithm or to the input data,
new HiC-Reg predictions were made by converting the corresponding hicprediction training sets to text files and using them for
training and prediction with HiC-Reg.
It is obvious from \autoref{fig:res:HicregVsHicprediction_recomputed} that the results did no longer differ much,
except for the value range. This indeed points to the input data as the main cause for the differences between hicprediction and HiC-Reg.

However, the computations from hicprediction were much more efficient both in terms of runtime and memory consumption,
see \autoref{tab:res:efficiency}. While training took almost three hours in HiC-Reg, it finished within around 90 seconds
in hicprediction -- and this even includes the time spent to split the data into five cross-validation sets and store them.
No efforts were made to find the cause for the significant runtime difference, 
but it was obvious that HiC-Reg was using only one CPU, while hicprediction was set up to use multithreading on
all available (four) CPUs for the random forest regressor.
\begin{table}[htb]
\resizebox{\textwidth}{!}{
\begin{tabular}{lcclcc}
\hline
                       & \multicolumn{2}{c}{\textbf{training}}                               &                      & \multicolumn{2}{c}{\textbf{prediction}}    
                         \\
                       & \multicolumn{1}{l}{runtime / min} & \multicolumn{1}{l}{memory / GB} &                      & \multicolumn{1}{l}{runtime / min} & 
\multicolumn{1}{l}{memory / GB} \\ \hline
\textbf{HiC-Reg}       & 175,58                            & 4,29                            & \multicolumn{1}{c}{} & 3,70                              & 1,92   
                         \\
\textbf{hicprediction} & 1,26                              & 0,70                            & \multicolumn{1}{c}{} & 1,18                              & 1,17   
                         \\ \hline
\end{tabular}}
\caption{computational effort HiC-Reg vs. hicprediction} \label{tab:res:efficiency}
\end{table}

For the example of CTCF in GM12878, the bedgraph input file converted from the corresponding BAM file 
proved qualitatively quite similar to the example file in the HiC-Reg github repository 
\cite{Roy2020}\footnote{full url: 
\url{https://github.com/Roy-lab/HiC-Reg/blob/master/Scripts/aggregateSignalInRegion/wgEncodeBroadHistoneGm12878CtcfStdRawDataRep1_chr17.counts}},
see \autoref{fig:app:GM12878:CTCF:convertedInput3MB} and \ref{fig:app:GM12878:CTCF:convertedInput100KB}.
However, the newly created file contained about \num{55000} (14.2\%) more lines than the example file, 
which suggests that the original input data to HiC-Reg must have been filtered in some way. 
The paper \cite{Zhang2019} provides general information on how the data have been processed,
e.\,g. with regard to software tools, but unfortunately lacks detailed information for example on specific filtering procedures or parameters.
Irrespective of this, predictions with HiC-Reg using the converted BAM files did not yield useful results, see 
\autoref{fig:res:HicregVsHicprediction_fromGroundUp}.
It could not be clarified what went wrong, 
but gradient predictions like these can occur when only distance is
considered as a feature, cf. \autoref{sec:improve:emptySamples}.
This could have happened either due to a bug in HiC-Reg 
or a misunderstanding of its input file formats, some of which are 
only specified by example.

A concluding discussion of all changes made to hicprediction within this masterproject
will be given in \autoref{sec:discussion:main}.
\begin{figure}[hbp]
\centering
\scriptsize
\import{figures/}{chr17-12-15_GM12878-on-K562_oursVsZhangPaper.pdf_tex}
\caption{HiC-Reg vs. hicprediction, GM12878 on K562, chromosome 17, \SI{5}{\kilo\bp}}
\label{fig:res:HicregVsHicprediction_supplementary}
\end{figure}
\begin{figure}[hbp]
\centering
\scriptsize
\import{figures/}{chr17-12-15_GM12878-on-K562_oursVsZhangPaper_noEmpty_sameInput.pdf_tex}
\caption{HiC-Reg vs. hicprediction, GM12878 on K562, chromosome 17, \SI{25}{\kilo\bp}, HiC-Reg input data converted from hicprediction datasets}
\label{fig:res:HicregVsHicprediction_recomputed}
\end{figure}
\begin{figure}[hbp]
\centering
\scriptsize
\import{figures/}{GM12878-on-K562_HicregFromGroundUp.pdf_tex}
\caption{HiC-Reg prediction from BAM files, GM12878 on K562, chromosome 17, \SI{5}{\kilo\bp}}
\label{fig:res:HicregVsHicprediction_fromGroundUp}
\end{figure}
\begin{figure}[hbp]
\centering
\begin{subfigure}[t]{.495\textwidth}
  \centering
  \resizebox{\textwidth}{!}{
  \scriptsize
 \import{figures/}{Pearson_GM12878-on-K562_hicregFromSupplementary_vs_hicprediction.pdf_tex}}
 \caption{HiC-Reg data from publication \cite{Zhang2019}}
 \label{fig:res:Pearson:K562:chr17:5kb:HicregFromPublication}
\end{subfigure}\hfill%
\begin{subfigure}[t]{.495\textwidth}
  \centering
  \resizebox{\textwidth}{!}{
  \scriptsize
 \import{figures/}{Pearson_GM12878-on-K562_noEmpty_HicregVsHicprediction_sameInputs.pdf_tex}}
 \caption{HiC-Reg input data converted from hicprediction datasets}
 \label{fig:res:Pearson:K562:chr17:5kb:HicregFromConversion}
\end{subfigure}
\begin{subfigure}{.495\textwidth}
  \vspace{5mm}
  \centering
  \resizebox{\textwidth}{!}{
  \scriptsize
 \import{figures/}{Pearson_GM12878-on-K562_HicregFromGroundUp.pdf_tex}}
 \caption{HiC-Reg input data computed from BAM files}
 \label{fig:res:Pearson:K562:chr17:5kb:HicregFromBam}
\end{subfigure}
\caption{Pearson correlation comparison HiC-Reg vs. hicprediction}
\label{fig:res:Pearson:hicRegVsHicprediction}
\end{figure}