\section{Prior Work}
Having introduced the fundamentals of both processes,
it shall now be shown how existing ChIP-seq- and Hi-C data 
can be used to to predict unknown Hi-C matrices.
The following subsections will thus give an overview of the
method developed by Andre Bajorat during his masterproject, hicprediction \cite{Bajorat2019},
and its foundation, HiC-Reg \cite{Zhang2019, Zhang2018}.
This is preceded by a brief introduction into random forests,
which are the common basis of both approaches.

\subsection{Random forest} \label{sec:prior:randomForests}
Random forests were conceived by Leo Breiman in 2001 \cite{Breiman2001} and are based on decision trees,
probably the most simple approach for supervised machine learning.

Decision trees take as input a set of samples with relevant features and then
``learn'' to predict a target feature by splitting the input dataset step by step, 
until the target feature can be decided on.
Here, only binary decision criteria on the training features are allowed (True/False), so the decision path can be
visualized in form of a tree, see \autoref{tab:prior:dataset:decisionTree} and \autoref{fig:prior:decisionTree}.
\begin{table}[hb]
\centering
\begin{tabular}{lccc} \hline
	   & \multicolumn{2}{c}{\textbf{train. features}} & \textbf{target feature}\\
 \textbf{sample nr.} & p0  	& p2 	& interact. count\\
 \hline
 0	& 0.0	& 0.0	& 0\\
 ...	& ...	& ...	& ...\\
 40000 	& 10.0  & 10.0	& 100\\
 40001 	& 10.0  & 0.0 	& 100\\
 40002  & 10.0  & 0.0 	& 100\\
 ...	& ...	& ...	& ...\\
 \hline
\end{tabular}
\caption{excerpt of a training dataset for a decision tree} \label{tab:prior:dataset:decisionTree}
\end{table}
\begin{figure}[hb]
 \centering
 \resizebox{0.75\textwidth}{!}{
 \import{figures/}{decisionTree.pdf_tex}}
 \caption{decision tree from hicprediction}
 \label{fig:prior:decisionTree}
\end{figure}

Within this masterproject, the target feature is Hi-C interaction count
and the training features are the signal values from ChIP-seq data of several proteins,
along with the genomic distance of the interacting regions.

Because decision trees are well known for overfitting, i.\,e. having a low bias but a very high variance,
random forests contain not only one, but generally a large number of decision trees.
Each tree is trained on a random subset of the original samples, potentially using random subsets
of the features for each splitting step.
For regression problems, the final prediction is then the average output of all trees, 
for classification problems usually the majority vote.
The rationale here is that random selection of the samples and features reduces the variance,
usually at the cost of increased bias. 
However, in most applications, the reduction in variance more than outweighs the increase in bias.

For regression trees, the cut points, e.\,g. ``p2 $\leq$ 5.0'', cf.\;\autoref{fig:prior:decisionTree}, 
are usually selected such that the variance in the remaining two datasets -- here, 
all samples with p2 $\leq$ 5.0 on the one hand and all samples with p2 $>$ 5.0 on
the other hand -- is reduced the most.
Searching for the optimal cut-point is computationally involved,
but meanwhile there are several open source implementations which handle the task efficiently.

Random forests have a number of tuning parameters. 
Probably the most important ones are the number of trees in the forest, 
the size of the subsample used to train each tree,
the maximum number of features to investigate for splitting
and the minimum number of samples allowed to remain in internal nodes, cf.\;\cite{scikitEnsemble2019}.
All of these are used by HiC-Reg and hicprediction, two random-forest-based methods
to predict unknown Hi-C matrices.

\subsection{HiC-Reg}
HiC-Reg was proposed by Shilu Zhang and colleagues at the Wisconsin Institute for Discovery in 2018 \cite{Zhang2019, Zhang2018}.
It takes known Hi-C matrices and ChIP-seq data as inputs to predict unknown Hi-C matrices using random forest regression.

HiC-reg accepts inputs as custom tab-separated text files \cite{Roy2020}.
For all pairs of (\SI{5000}{\kilo\bp})-regions within a size-adjustable window, the average
ChIP-seq read counts of all bins within the window and the distance itself 
are computed and a so-called \textsc{window}-dataset is formed from the valid pairs, see \autoref{fig:priorW:hicreg2}, upper left and middle.
Such a dataset and corresponding Hi-C interaction counts are then merged into a training set
for a custom random forest regressor. 

After training the regressor, predicting Hi-C interaction counts just requires feeding a test dataset, 
i.\,e. processed ChIP-seq data for the target cell line, into the trained random forest, 
which yields predicted interaction counts as output, \autoref{fig:priorW:hicreg2}, middle left. 
The results, too, are stored in a custom tab-separated text format.
\begin{figure}[p]
\centering
  \resizebox{!}{0.7\textheight}{
  \import{figures/}{hicreg5.pdf_tex}}
 \caption{HiC-Reg principle - \textsc{window} prediction}
 \label{fig:priorW:hicreg2}
\end{figure}
\begin{figure}[p]
\tiny
\centering
 \import{figures/}{zhang2019_predictions.pdf_tex}
 \caption[Hi-C matrices predicted by HiC-Reg]{Hi-C matrices predicted by HiC-Reg in \textsc{multi-cell} mode, cf.\,\cite[p.\,9]{Zhang2019}}
 \label{fig:priorW:hicregPredictions}
\end{figure}

In case data from multiple cell lines is available for training,
the window features of all training cell lines can be concatenated into so-called \textsc{multi-cell} test sets, 
cf.\,\cite[Fig. 1]{Zhang2019}, but these were not used throughout this masterproject.
By default, HiC-Reg is using five-fold cross validation on training sets, 
so it actually trains five different models for each training cell line and chromosome.
These models can then be used to predict interactions for a different
chromosome in the same cell line (\textsc{cross-chromosome}) or to predict interactions for the same chromosome in 
another cell line (\textsc{cross-cell}) -- which is more relevant for practical use cases of HiC-Reg.
The final predicted matrix is just the average output of the five models.

Despite the aim of predicting Hi-C interaction counts, the publication by Zhang et al. unfortunately contains only two small plots 
of actual Hi-C matrix snippets, depicted in their original size in \autoref{fig:priorW:hicregPredictions}, 
and instead relies on statistical measures to assess the quality of the predictions.
To this end, the two main metrics used are distance stratified Pearson 
correlation and the corresponding area under the correlation curve, \textsc{auc}.
While this choice is in line with other relevant publications in the field \cite{Pierro2017, Farre2018, Schwessinger2019},
it seems unfavorable to rely on Pearson correlation too much, since the correlation can be high for Hi-C matrices
of biologically unrelated origin \cite{Yang2017}. 
Within this report, correlation plots are thus always used in combination with plots of Hi-C matrix snippets.

Most of HiC-Reg, including the custom random forest regression model, is implemented in C++, 
complemented by python-, shell- and matlab scripts e.\,g. for converting and concatenating input data or 
computing various metrics. The source code and most binaries are available from github \cite{Roy2020}.

\subsection{Hicprediction}
Hicprediction was conceived and implemented as part of Andre Bajorat's masterproject at the Bioinformatics Group of the University of Freiburg
in 2019 \cite{Bajorat2019}. It is basically a python implementation of the ideas of Zhang et al. outlined above, 
yet with a few technical differences.

In distinction from HiC-Reg, hicprediction accepts the common file formats narrowPeak and cooler \cite{Abdennur2019}
as inputs for ChIP-seq and Hi-C data, respectively, and also outputs Hi-C matrices directly in cooler format.
This eliminates the need for text format conversions, 
but also limits applicability to cases where appropriate input files are available.
Another major difference to HiC-Reg is that hicprediction does not perform cross-validation on the training data
for \textsc{cross-cell} and \textsc{cross-chromosome} predictions.
Instead, it always trains its random forest on the full training set, 
employing HiC-Reg's \textsc{window} approach described above.
With regard to binning proteins and computing \textsc{window}-features, 
hicprediction supports more aggregation methods than HiC-Reg, mean (``avg''), sum and max, see \autoref{fig:priorW:hicpredFlow}.
While this might be interesting for future research, sum and max have not been used throughout this masterproject to
maintain comparability with the results from Zhang and colleagues.

Hicprediction is made up from four main python modules, which basically implement the workflow depicted in 
\autoref{fig:priorW:hicreg2}. 
\begin{figure}[htb]
 \small
 \centering
 \resizebox{\textwidth}{!}{
 \import{figures/}{hicprediction_flow1.pdf_tex}}
 \caption{Flow diagram of hicprediction (cross-chromosome prediction)}
 \label{fig:priorW:hicpredFlow}
\end{figure}
First, \emph{createBasefile.py} \cite{hicpred2020bf} aggregates the protein data for training and test into bins of fixed width and stores them in an h5py file, 
see 
\autoref{fig:priorW:hicpredFlow}, upper and lower left.
Additionally, input Hi-C matrices are cut into separate matrices, one per chromosome, and stored for further usage.
Next, \emph{createTrainingSet.py} \cite{hicpred2020cts} joins binned protein data with Hi-C interaction counts, computes the \textsc{window}-features mentioned 
above and 
stores the resulting datasets in a compressed binary format (\autoref{fig:priorW:hicpredFlow}, right).
The training dataset is then used by \emph{training.py} \cite{hicpred2020tr} to train a scikit-learn random forest regressor \cite{Pedregosa2011}, 
which is again stored in binary format (\autoref{fig:priorW:hicpredFlow}, middle).
Finally, \emph{predict.py} \cite{hicpred2020pr} takes the test set with protein data from the target cell line, 
runs it through the trained model and outputs a predicted Hi-C matrix in cooler format (\autoref{fig:priorW:hicpredFlow}, middle left).
Additionally, a tab-separated text file can be output, containing distance stratified Pearson correlation values and \textsc{auc},
plus some other statistical measures and parameter settings for reference. 

The original version of hicprediction is available from github \cite{BajoratHicpredWWW2019} and can be 
installed using conda \cite{minicondaWWW2019}.
Each of the four python modules mentioned above takes a number of options and further inputs, 
which are described in detail on the readme page in github.